\chapter[Causes Of PHL]{Contributory factors to \acrshort{phl} in Nigeria}
\label{cp:causes-of-phl}

{
\parindent0pt

%\textit{Authors: Edun Joshua, Mbuotidem Awak, Makinde Kayode}

% \textit{Current Version: 2.2.4}

%\textit{License: \LaTeX~Project Public License v1.3c}

%\textit{Official Repository: \href{https://github.com/joseareia/ipleiria-thesis}{GitHub Repository}}

\vspace{.935em}
 This chapter examines the primary factors driving Nigeria’s post-harvest losses – poor transportation infrastructure, inadequate storage and cold-chain capacity, and limited market access – and explains how each contributes to the problem.

 \section{Poor Transportation Infrastructure}
Nigeria’s road and transport network is widely cited as a major culprit in post-harvest loss. Most rural roads are unpaved and poorly maintained, making travel slow, costly, and unreliable \citep{PinnacleTimes2024, VoiceofNigeria2023}. In fact, about 80\% of Nigerian feeder roads are unpaved and become impassable during the rainy season \citep{PinnacleTimes2024}. This forces farmers to make long, bumpy trips to market on motorcycle or in old trucks. One maize farmer in Osun State lamented that after harvesting, “we have no means of transporting [products] to urban centres…we eat what we can while a good percentage wastes away” \citep{PinnacleTimes2024}.
\\

Bad roads and inadequate vehicles increase both transit time and physical damage to produce. Perishable crops (tomatoes, peppers, leafy vegetables) begin to spoil or get bruised during long journeys in hot, open trucks. In one reported case, a Kano tomato farmer drove 14 hours in a non–air-conditioned van on a 32°C day; by the time his produce reached market, the tomatoes were already “rotting” and often thrown away unsold \citep{Ikegwuonu2018}.
Such delays and heat stress can reduce shelf life from days to hours. Even for storable staples (maize, rice, yams), poor roads mean farmers often wait too long before selling; delays invite pests (weevils, rodents) and moisture damage during transport.
\\

This infrastructure gap has been quantified in national studies. For example, the Food and Agriculture Organization (FAO) estimates that up to 40\% of Nigeria’s agricultural output is lost post-harvest due to inadequate infrastructure \citep{PinnacleTimes2024}. Government officials also emphasize that improving rural roads can directly cut losses: according to Nigeria’s Rural Access Project coordinator, better feeder roads and bridge repairs would “reduce post-harvest losses, cost of transportation and accident rates” for farmers citep{VoiceofNigeria2023}.
\\

In short, the combination of long distances, rough roads and few refrigerated trucks means a large share of freshly-harvested crops never reach markets in good condition.

\subsection{Key effects of poor transportation}

\begin{description}
    \item[Delays and spoilage:] Long travel times on bad roads (often on footpaths or seasonal tracks) expose produce to heat, moisture and pests, so fruits and vegetables rot en route \citep{Ikegwuonu2018}.
    
    \item[Physical damage] Bumpy rides in unpadded vehicles bruise grains and produce, leading to quality losses. Crushed bags attract pests and mold.
    
    \item[High costs and leakage] Weak logistics force farmers to pay steep transport fees, effectively shrinking their margins. Some give up on distant markets altogether, dumping excess harvest near the farm.
\end{description}


Together, these factors mean that many Nigerian farmers see double-digit percentage losses just due to transit. For instance, data from the African Postharvest Losses Information System (APHLIS) shows ~17\% of Nigeria's maize is lost from harvest through delivery \citet{APHLIS}- and that figure would be higher on poor roads.


\subsection{Inadequate Cold Storage and Processing}
Another root cause of Nigeria’s post-harvest losses is the near-absence of cold-chain and storage facilities. Most Nigerian farmers and traders have no access to refrigerated warehouses or cooling trucks. After harvest, highly perishable goods like tomatoes, fruits, vegetables, dairy and meat immediately begin to deteriorate without temperature control \citep{OTACCWA2025}.
\vspace{\myvspace}

Nigeria’s climate and power constraints make this especially severe. In most rural areas, electricity is intermittent or unavailable, and conventional refrigerators require too much power. As one report notes, “unlike the United States and Europe…cold refrigeration is virtually nonexistent in [Nigerian] farms and marketplaces” \citep{Ikegwuonu2018}.
\vspace{\myvspace}

The impact is dramatic: one analysis found that about 45\% of Nigeria’s perishable produce spoils at some point after harvest solely because of lack of cold storage\citep{Ikegwuonu2018}. Without refrigeration, simple transport times of a day or two can be fatal. For example, in the IFPRI ColdHubs case study above, tomatoes lost roughly two-thirds of their market value by afternoon due to heat and rot \citep{Ikegwuonu2018}. Similarly, fruits like mangoes or watermelons will ferment or shrivel if not cooled within hours. Even root crops (cassava, yams) and grains lose quality if stored too long without drying or cooling; high humidity in sacks leads to mold and aflatoxins.
\vspace{\myvspace}

Moreover, Nigeria lacks processing and value-addition facilities that could reduce spoilage. For instance, small-scale drying kilns, canning plants, or flour mills are rare in many regions. If farmers had accessible rice mills or tomato canneries nearby, they could convert part of their harvest to shelf-stable products, greatly extending shelf life. In reality, most produce must be sold raw. This mismatch means that whenever markets glutted after harvest, farmers often have no choice but to sell quickly at low prices or waste the excess.
\vspace{\myvspace}

In sum, the lack of a modern cold chain (from farm to market) means Nigeria forfeits nearly half its high-value harvest to heat and spoilage
\citep{Ikegwuonu2018, OTACCWA2025}. Experts note that “a modern cold chain system, combined with improved infrastructure and logistics, will be key to mitigating these losses” \citep{Ikegwuonu2018, OTACCWA2025}. As it stands, perishables often never reach peak freshness, and farmers’ incomes suffer accordingly.

\subsection{Limited Market Access and Information}
A related factor is how market barriers amplify post-harvest losses. Many smallholder farmers in Nigeria are geographically or socially isolated from buyers. Poor linkages mean harvests cannot be sold quickly and efficiently. Key issues include:

\begin{description}
\item[Fragmented production:] Nigerian farms are small and scattered. Most farmers sell independently in local village markets. Without aggregation, they face high transaction costs transporting small loads, and cannot negotiate strong prices. The \citet{FarmSupportSolutions2025} notes that fragmented production and inconsistent quality “reduces market acceptance” and makes rural supply chains inefficient. When multiple small producers simultaneously bring the same crop to town, local prices crash and some output cannot be sold at all.

\item[Lack of price and demand information:] Many farmers do not have real-time data on market prices or demand trends. Without mobile market services or cooperatives, they often sell based on gut feeling. This information gap means farmers miss opportunities to time their sales for higher prices or send produce where demand is growing. In practice, it leads to gluts in some markets (driving prices down and unsold stacks to waste) while other regions suffer shortages. As FSSS reports, “limited market information…leads to low economies of scale” and exacerbates waste \citep{FarmSupportSolutions2025}.

\item[Inefficient value chains:] Poor market access also results from missing infrastructure and services. For example, contract storage (warehouse receipt systems) or cold trucks for linking farmers to distant processors are largely unavailable. Without these, farmers often face long waits or forced distress sales. Research in Nigeria has noted that inadequate marketing systems and governance gaps in distribution (such as lack of financing for storage) are key socio-economic causes of loss \citep{Ogundele2022}.
\end{description}

In other words, even if crops survive transport, market inefficiencies can leave them unsold. For instance, one survey found that almost half of smallholder farmers were unable to sell their preferred quantity because of volatile prices and a lack of buyers
\vspace{\myvspace}

Addressing market access could dramatically cut losses: digital marketplaces and aggregation hubs are emerging as solutions. By connecting farmers directly with buyers and sharing price information, such platforms reduce the risk of unsold crop. For example, pilot programs like Foodstuff Store and Soluta leverage mobile apps to link remote farmers with city consumers and traders, helping ensure produce finds a market quickly \citep{Tenebe2024, FarmSupportSolutions2025}. These innovations hint at how improved market linkages can complement transport and storage solutions.

\subsection{Conclusion}

Nigeria’s high post-harvest losses have clear, practical causes rooted in infrastructure and market systems. In summary: crumbling rural roads and transport mean crops spoil before sale
\citep{PinnacleTimes2024, VoiceofNigeria2023}
; absence of cold storage and processing leaves perishables to rot or shrivel
\citep{Ikegwuonu2018, OTACCWA2025}
; and poor market integration forces farmers into inefficient, wasteful sales patterns
\citep{FarmSupportSolutions2025}. These problems are interlinked: for example, even if transport is available, the benefit is lost if there is nowhere to store or sell the produce at proper prices. Studies emphasize that tackling Nigeria’s PHL requires a multifaceted approach. The challenges are “complex and interrelated,” so solutions must combine technical fixes with economic and policy reforms
\citep{Ikegwuonu2018, Abulude2024}.
\vspace{\myvspace}

In practice, this means investing in all links of the chain: paving and maintaining feeder roads and bridges, expanding clean energy for cold storage, supporting local processing centers, and empowering farmers with market data and aggregation mechanisms. According to experts, “a modern cold chain system, combined with improved infrastructure and logistics, will be key to mitigating these losses”\citep{Ikegwuonu2018}.
\vspace{\myvspace}

In conclusion, reducing Nigeria’s PHL burden (currently on the order of tens of billions of naira per year) hinges on strengthening the transportation-storage-market triad. Hackathon innovations that create digital marketplaces or logistics solutions directly address this nexus. By helping farmers find buyers and manage their produce more efficiently, such tools can cut waste and preserve value. In doing so, they support food security and farmers’ incomes – goals that depend squarely on fixing the infrastructure and informational gaps highlighted above.
}